\documentclass{article}
\usepackage{amsmath}
\usepackage{amssymb}
\usepackage{geometry}
\geometry{margin=1in}

\title{Discrete Update for 2D Damped Wave Equation}
\author{AudioRipple Project}
\date{}

\begin{document}

\maketitle

\section*{Continuous PDE}

\begin{equation}
\frac{\partial^2 Z}{\partial t^2}
= c^2 \nabla^2 Z - \gamma \frac{\partial Z}{\partial t},
\end{equation}
where $\nabla^2 Z$ is the Laplacian operator defined as
\begin{equation}
    \nabla^2 Z = \frac{\partial^2 Z}{\partial x^2} + \frac{\partial^2 Z}{\partial y^2}.
\end{equation}

\section*{Approximate Time Derivatives}

\subsection*{Second time derivative}

\begin{equation}
\frac{\partial^2 Z}{\partial t^2}
\approx
\frac{Z^{n+1}_{i,j} - 2Z^n_{i,j} + Z^{n-1}_{i,j}}{\Delta t^2}.
\end{equation}

\subsection*{First time derivative (damping)}

\begin{equation}
\frac{\partial Z}{\partial t}
\approx
\frac{Z^n_{i,j} - Z^{n-1}_{i,j}}{\Delta t}.
\end{equation}

\section*{Approximate Spatial Derivatives (Laplacian)}

\subsection*{In x}

\begin{equation}
\frac{\partial^2 Z}{\partial x^2}
\approx
\frac{Z_{i+1,j} - 2Z_{i,j} + Z_{i-1,j}}{(\Delta x)^2}.
\end{equation}

\subsection*{In y}

\begin{equation}
\frac{\partial^2 Z}{\partial y^2}
\approx
\frac{Z_{i,j+1} - 2Z_{i,j} + Z_{i,j-1}}{(\Delta y)^2}.
\end{equation}

\subsection*{Combined 2D Laplacian (five-point stencil)}

\begin{equation}
\nabla^2 Z_{i,j}
\approx
\frac{Z_{i+1,j} + Z_{i-1,j} + Z_{i,j+1} + Z_{i,j-1} - 4Z_{i,j}}{(\Delta x)^2}.
\end{equation}

\newpage

\section*{Solve for Next Time Value}

\begin{equation}
\frac{Z^{n+1}_{i,j} - 2Z^n_{i,j} + Z^{n-1}_{i,j}}{\Delta t^2}
= c^2 \nabla^2 Z^n_{i,j} - \gamma \frac{Z^n_{i,j} - Z^{n-1}_{i,j}}{\Delta t}.
\end{equation}

\[
\Longrightarrow
\]

\begin{equation}
\begin{aligned}
Z^{n+1}_{i,j}
&= 2Z^n_{i,j}
- Z^{n-1}_{i,j}
+ \left(\frac{c \Delta t}{\Delta x}\right)^2 \bigl(Z_{i+1,j} + Z_{i-1,j} + Z_{i,j+1} + Z_{i,j-1} - 4Z_{i,j}\bigr) \\
&\quad - \gamma \Delta t \,(Z^n_{i,j} - Z^{n-1}_{i,j}).
\end{aligned}
\end{equation}

\section*{Code Expression (Final Form)}

\begin{equation}
\begin{aligned}
Z_{\text{new}}
&= \underbrace{2 Z - Z_{\text{old}}}_{\text{leap-frog}}
+ \underbrace{c2\_dt2 \cdot \text{laplacian}(Z)}_{\text{curvature}}
- \underbrace{(1 - \text{damping}) \cdot \Delta t \cdot (Z - Z_{\text{old}})}_{\text{damping correction}}, \\
c2\_dt2 &= \left(\frac{c \Delta t}{\Delta x}\right)^2.
\end{aligned}
\end{equation}

\section*{Stability Condition}

\begin{equation}
\frac{c \Delta t}{\Delta x} \le \frac{1}{\sqrt{2}}.
\end{equation}

% --------------------------------------------------------------
\newpage
\section*{Boundary Conditions}
\label{sec:bc-detailed}

Finite-difference solvers need a rule for every “missing neighbour”
that lies outside the computational grid.  Four families cover most
practical situations:

\subsection*{1. Periodic (wrap-around)}

Neighbour indices are taken modulo the grid size, so waves leaving one
edge instantly re-enter from the opposite edge.  Energy is conserved and
no reflection occurs.

\subsection*{2. Dirichlet (fixed value)}

We require
\[
Z = Z_\mathrm{bnd}\quad (\text{often } Z_\mathrm{bnd}=0).
\]
Physically this models a string clamped at its end or an acoustic
pressure-release surface.  Waves reflect with \emph{unit amplitude} and a
\emph{\(180^\circ\) phase flip}.  No energy is absorbed unless additional
damping terms are added.

\subsection*{3. Neumann (zero-gradient)}

We impose
\[
\partial_{\mathbf n}Z = 0,
\]
i.e.\ the normal derivative vanishes.  This represents a perfectly rigid
wall (hard boundary) or a free string end.  Waves reflect with unit
amplitude and \emph{no} phase flip.

\subsection*{4. Absorbing / radiative}

To let energy leave the domain, we add a spatially varying damping term
\(-\sigma(\mathbf x)\,Z\), often chosen to increase smoothly toward the
grid edge (‘‘sponge layer’’) or derived from a perfectly matched layer
(PML).  \(\sigma\) can itself be frequency-dependent to model materials
that absorb high frequencies more strongly than low.

\bigskip
\noindent
\textbf{Unified view.}  All four cases can be expressed with the mask-based
Laplacian of Section~\ref{sec:mask-bc}: set edge weights and (optionally)
diagonal damping so the discrete operator already \emph{knows} the
boundary behaviour—no per-step if-statements are needed.
% --------------------------------------------------------------

% --------------------------------------------------------------
\newpage
\section*{Internal Reflective (and Absorbing) Interfaces}
\label{sec:mask-bc}

Suppose we embed an arbitrary interface (e.g.\ a circle) inside a
rectangular grid.  A binary mask
\[
M_{ij}\;=\;\begin{cases}
1, & \text{cell lies on the interface},\\
0, & \text{otherwise}
\end{cases}
\]
marks those locations.

\subsection*{1. Continuous boundary statements}

\begin{itemize}
  \item \textbf{Dirichlet:} \(Z=0\) at the interface  
        (soft wall, pressure release).  Reflection coefficient
        \(R=-1\) (full magnitude, phase inversion).
  \item \textbf{Neumann:} \(\partial_{\mathbf n}Z=0\)  
        (rigid wall).  Reflection coefficient \(R=+1\).
\end{itemize}

\subsection*{2. Why the ghost values are $-Z$ or $+Z$}

Consider a one-dimensional grid with points \(\dots,\,i-1,\,i,\,i+1,\dots\)
where \(i\) is just \emph{inside} the interface.

\begin{enumerate}
\item The second derivative at \(i\) needs \(Z_{i+1}\), which we do not
      store.  We invent a \emph{ghost value} \(Z^\ast_{i+1}\).
\item For a Dirichlet wall we want the physical displacement to be zero
      \emph{exactly halfway} between \(i\) and \(i+1\).  Extending the
      function \emph{oddly} (\(Z(-x)=-Z(x)\)) guarantees that midpoint
      is zero.  Hence
      \[
        Z^\ast_{i+1} = -\,Z_i.
      \]
\item For a Neumann wall we want zero slope, so we extend the function
      \emph{evenly} (\(Z(-x)=Z(x)\)), giving
      \[
        Z^\ast_{i+1} = Z_i.
      \]
\end{enumerate}

\paragraph{Stencil check (1-D)}

\[
\frac{Z^\ast_{i+1}-2Z_i+Z_{i-1}}{(\Delta x)^2}=
\begin{cases}
\dfrac{-3Z_i+Z_{i-1}}{(\Delta x)^2}, & \text{Dirichlet},\\[6pt]
\dfrac{-Z_i+Z_{i-1}}{(\Delta x)^2},  & \text{Neumann}.
\end{cases}
\]
Both choices recover the correct reflective behaviour in leap-frog
time stepping.

\subsection*{3. Masked Laplacian (graph view)}

Define edge indicators
\(\eta^{\uparrow}_{ij},\eta^{\downarrow}_{ij},
 \eta^{\leftarrow}_{ij},\eta^{\rightarrow}_{ij}\in\{0,1\}\)
that equal 1 when the neighbour is \emph{not} masked.
Let \(d_{ij}=\sum\eta\).  Then
\[
\nabla^2_{\text{mask}}Z_{ij}=
\bigl(
  \eta^{\uparrow}Z_{i-1,j}+\eta^{\downarrow}Z_{i+1,j}
 +\eta^{\leftarrow}Z_{i,j-1}+\eta^{\rightarrow}Z_{i,j+1}
\bigr)
-
\begin{cases}
d_{ij}\,Z_{ij}, & \text{Neumann},\\[6pt]
4\,Z_{ij}, & \text{Dirichlet}.
\end{cases}
\]
Setting the diagonal to 4 in the Dirichlet case mimics the
\(Z^\ast=-Z\) rule while keeping the operator symmetric.

\subsection*{4. Partial transmission (hybrid interface)}

We can interpolate continuously between full reflection and full
transmission by scaling each edge that crosses the interface:

\[
w\,\in\,[0,1],\quad
R=\frac{1-w}{1+w},\qquad
w=\frac{1-R}{1+R}.
\]

\begin{itemize}
  \item \(w=0\;\Rightarrow\;R=1\): perfect mirror.
  \item \(w=1\;\Rightarrow\;R=0\): transparent membrane.
  \item \(0<w<1\): partial reflection \(\bigl(|R|<1\bigr)\).
\end{itemize}

Implementationally we multiply the adjacency entries that cross the
mask by \(w\).

\subsection*{5. Absorbing (damped) interface}

If a portion of the wave should disappear inside the interface (e.g.,
porous absorber), add a local damping term
\(-\sigma_{ij}\,Z_{ij}\) with \(\sigma_{ij}>0\).
Frequency-dependent absorption can be modelled by letting
\(\sigma\) (or \(w\)) depend on frequency—a topic beyond this primer.

\paragraph{Stability.}  Because the modified Laplacian remains symmetric
positive semi-definite (and damping is non-negative), the standard
CFL limit
\[
c\,\Delta t/\Delta x \;\le\; 1/\sqrt{2}
\]
still guarantees stability.

% --------------------------------------------------------------

% =====================================================================

\section*{Graph Laplacian Viewpoint}

The discrete Laplacian matrix \(L\) on a regular 2D grid with 4-connected neighbours corresponds to

\begin{equation}
L Z = -4Z_{i,j} + Z_{i+1,j} + Z_{i-1,j} + Z_{i,j+1} + Z_{i,j-1}.
\end{equation}

Which in matrix form can be expressed as
\begin{equation}
    L = \begin{bmatrix}
        -4 & 1 & 0 & 0 & \cdots \\
        1 & -4 & 1 & 0 & \cdots \\
        0 & 1 & -4 & 1 & \cdots \\
        0 & 0 & 1 & -4 & \cdots \\
        \vdots & \vdots & \vdots & \vdots & \ddots
    \end{bmatrix}.
\end{equation}

This is exactly the five-point stencil.

\end{document}

